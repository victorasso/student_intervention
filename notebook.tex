
% Default to the notebook output style

    


% Inherit from the specified cell style.




    
\documentclass[11pt]{article}

    
    
    \usepackage[T1]{fontenc}
    % Nicer default font (+ math font) than Computer Modern for most use cases
    \usepackage{mathpazo}

    % Basic figure setup, for now with no caption control since it's done
    % automatically by Pandoc (which extracts ![](path) syntax from Markdown).
    \usepackage{graphicx}
    % We will generate all images so they have a width \maxwidth. This means
    % that they will get their normal width if they fit onto the page, but
    % are scaled down if they would overflow the margins.
    \makeatletter
    \def\maxwidth{\ifdim\Gin@nat@width>\linewidth\linewidth
    \else\Gin@nat@width\fi}
    \makeatother
    \let\Oldincludegraphics\includegraphics
    % Set max figure width to be 80% of text width, for now hardcoded.
    \renewcommand{\includegraphics}[1]{\Oldincludegraphics[width=.8\maxwidth]{#1}}
    % Ensure that by default, figures have no caption (until we provide a
    % proper Figure object with a Caption API and a way to capture that
    % in the conversion process - todo).
    \usepackage{caption}
    \DeclareCaptionLabelFormat{nolabel}{}
    \captionsetup{labelformat=nolabel}

    \usepackage{adjustbox} % Used to constrain images to a maximum size 
    \usepackage{xcolor} % Allow colors to be defined
    \usepackage{enumerate} % Needed for markdown enumerations to work
    \usepackage{geometry} % Used to adjust the document margins
    \usepackage{amsmath} % Equations
    \usepackage{amssymb} % Equations
    \usepackage{textcomp} % defines textquotesingle
    % Hack from http://tex.stackexchange.com/a/47451/13684:
    \AtBeginDocument{%
        \def\PYZsq{\textquotesingle}% Upright quotes in Pygmentized code
    }
    \usepackage{upquote} % Upright quotes for verbatim code
    \usepackage{eurosym} % defines \euro
    \usepackage[mathletters]{ucs} % Extended unicode (utf-8) support
    \usepackage[utf8x]{inputenc} % Allow utf-8 characters in the tex document
    \usepackage{fancyvrb} % verbatim replacement that allows latex
    \usepackage{grffile} % extends the file name processing of package graphics 
                         % to support a larger range 
    % The hyperref package gives us a pdf with properly built
    % internal navigation ('pdf bookmarks' for the table of contents,
    % internal cross-reference links, web links for URLs, etc.)
    \usepackage{hyperref}
    \usepackage{longtable} % longtable support required by pandoc >1.10
    \usepackage{booktabs}  % table support for pandoc > 1.12.2
    \usepackage[inline]{enumitem} % IRkernel/repr support (it uses the enumerate* environment)
    \usepackage[normalem]{ulem} % ulem is needed to support strikethroughs (\sout)
                                % normalem makes italics be italics, not underlines
    

    
    
    % Colors for the hyperref package
    \definecolor{urlcolor}{rgb}{0,.145,.698}
    \definecolor{linkcolor}{rgb}{.71,0.21,0.01}
    \definecolor{citecolor}{rgb}{.12,.54,.11}

    % ANSI colors
    \definecolor{ansi-black}{HTML}{3E424D}
    \definecolor{ansi-black-intense}{HTML}{282C36}
    \definecolor{ansi-red}{HTML}{E75C58}
    \definecolor{ansi-red-intense}{HTML}{B22B31}
    \definecolor{ansi-green}{HTML}{00A250}
    \definecolor{ansi-green-intense}{HTML}{007427}
    \definecolor{ansi-yellow}{HTML}{DDB62B}
    \definecolor{ansi-yellow-intense}{HTML}{B27D12}
    \definecolor{ansi-blue}{HTML}{208FFB}
    \definecolor{ansi-blue-intense}{HTML}{0065CA}
    \definecolor{ansi-magenta}{HTML}{D160C4}
    \definecolor{ansi-magenta-intense}{HTML}{A03196}
    \definecolor{ansi-cyan}{HTML}{60C6C8}
    \definecolor{ansi-cyan-intense}{HTML}{258F8F}
    \definecolor{ansi-white}{HTML}{C5C1B4}
    \definecolor{ansi-white-intense}{HTML}{A1A6B2}

    % commands and environments needed by pandoc snippets
    % extracted from the output of `pandoc -s`
    \providecommand{\tightlist}{%
      \setlength{\itemsep}{0pt}\setlength{\parskip}{0pt}}
    \DefineVerbatimEnvironment{Highlighting}{Verbatim}{commandchars=\\\{\}}
    % Add ',fontsize=\small' for more characters per line
    \newenvironment{Shaded}{}{}
    \newcommand{\KeywordTok}[1]{\textcolor[rgb]{0.00,0.44,0.13}{\textbf{{#1}}}}
    \newcommand{\DataTypeTok}[1]{\textcolor[rgb]{0.56,0.13,0.00}{{#1}}}
    \newcommand{\DecValTok}[1]{\textcolor[rgb]{0.25,0.63,0.44}{{#1}}}
    \newcommand{\BaseNTok}[1]{\textcolor[rgb]{0.25,0.63,0.44}{{#1}}}
    \newcommand{\FloatTok}[1]{\textcolor[rgb]{0.25,0.63,0.44}{{#1}}}
    \newcommand{\CharTok}[1]{\textcolor[rgb]{0.25,0.44,0.63}{{#1}}}
    \newcommand{\StringTok}[1]{\textcolor[rgb]{0.25,0.44,0.63}{{#1}}}
    \newcommand{\CommentTok}[1]{\textcolor[rgb]{0.38,0.63,0.69}{\textit{{#1}}}}
    \newcommand{\OtherTok}[1]{\textcolor[rgb]{0.00,0.44,0.13}{{#1}}}
    \newcommand{\AlertTok}[1]{\textcolor[rgb]{1.00,0.00,0.00}{\textbf{{#1}}}}
    \newcommand{\FunctionTok}[1]{\textcolor[rgb]{0.02,0.16,0.49}{{#1}}}
    \newcommand{\RegionMarkerTok}[1]{{#1}}
    \newcommand{\ErrorTok}[1]{\textcolor[rgb]{1.00,0.00,0.00}{\textbf{{#1}}}}
    \newcommand{\NormalTok}[1]{{#1}}
    
    % Additional commands for more recent versions of Pandoc
    \newcommand{\ConstantTok}[1]{\textcolor[rgb]{0.53,0.00,0.00}{{#1}}}
    \newcommand{\SpecialCharTok}[1]{\textcolor[rgb]{0.25,0.44,0.63}{{#1}}}
    \newcommand{\VerbatimStringTok}[1]{\textcolor[rgb]{0.25,0.44,0.63}{{#1}}}
    \newcommand{\SpecialStringTok}[1]{\textcolor[rgb]{0.73,0.40,0.53}{{#1}}}
    \newcommand{\ImportTok}[1]{{#1}}
    \newcommand{\DocumentationTok}[1]{\textcolor[rgb]{0.73,0.13,0.13}{\textit{{#1}}}}
    \newcommand{\AnnotationTok}[1]{\textcolor[rgb]{0.38,0.63,0.69}{\textbf{\textit{{#1}}}}}
    \newcommand{\CommentVarTok}[1]{\textcolor[rgb]{0.38,0.63,0.69}{\textbf{\textit{{#1}}}}}
    \newcommand{\VariableTok}[1]{\textcolor[rgb]{0.10,0.09,0.49}{{#1}}}
    \newcommand{\ControlFlowTok}[1]{\textcolor[rgb]{0.00,0.44,0.13}{\textbf{{#1}}}}
    \newcommand{\OperatorTok}[1]{\textcolor[rgb]{0.40,0.40,0.40}{{#1}}}
    \newcommand{\BuiltInTok}[1]{{#1}}
    \newcommand{\ExtensionTok}[1]{{#1}}
    \newcommand{\PreprocessorTok}[1]{\textcolor[rgb]{0.74,0.48,0.00}{{#1}}}
    \newcommand{\AttributeTok}[1]{\textcolor[rgb]{0.49,0.56,0.16}{{#1}}}
    \newcommand{\InformationTok}[1]{\textcolor[rgb]{0.38,0.63,0.69}{\textbf{\textit{{#1}}}}}
    \newcommand{\WarningTok}[1]{\textcolor[rgb]{0.38,0.63,0.69}{\textbf{\textit{{#1}}}}}
    
    
    % Define a nice break command that doesn't care if a line doesn't already
    % exist.
    \def\br{\hspace*{\fill} \\* }
    % Math Jax compatability definitions
    \def\gt{>}
    \def\lt{<}
    % Document parameters
    \title{student\_intervention}
    
    
    

    % Pygments definitions
    
\makeatletter
\def\PY@reset{\let\PY@it=\relax \let\PY@bf=\relax%
    \let\PY@ul=\relax \let\PY@tc=\relax%
    \let\PY@bc=\relax \let\PY@ff=\relax}
\def\PY@tok#1{\csname PY@tok@#1\endcsname}
\def\PY@toks#1+{\ifx\relax#1\empty\else%
    \PY@tok{#1}\expandafter\PY@toks\fi}
\def\PY@do#1{\PY@bc{\PY@tc{\PY@ul{%
    \PY@it{\PY@bf{\PY@ff{#1}}}}}}}
\def\PY#1#2{\PY@reset\PY@toks#1+\relax+\PY@do{#2}}

\expandafter\def\csname PY@tok@w\endcsname{\def\PY@tc##1{\textcolor[rgb]{0.73,0.73,0.73}{##1}}}
\expandafter\def\csname PY@tok@c\endcsname{\let\PY@it=\textit\def\PY@tc##1{\textcolor[rgb]{0.25,0.50,0.50}{##1}}}
\expandafter\def\csname PY@tok@cp\endcsname{\def\PY@tc##1{\textcolor[rgb]{0.74,0.48,0.00}{##1}}}
\expandafter\def\csname PY@tok@k\endcsname{\let\PY@bf=\textbf\def\PY@tc##1{\textcolor[rgb]{0.00,0.50,0.00}{##1}}}
\expandafter\def\csname PY@tok@kp\endcsname{\def\PY@tc##1{\textcolor[rgb]{0.00,0.50,0.00}{##1}}}
\expandafter\def\csname PY@tok@kt\endcsname{\def\PY@tc##1{\textcolor[rgb]{0.69,0.00,0.25}{##1}}}
\expandafter\def\csname PY@tok@o\endcsname{\def\PY@tc##1{\textcolor[rgb]{0.40,0.40,0.40}{##1}}}
\expandafter\def\csname PY@tok@ow\endcsname{\let\PY@bf=\textbf\def\PY@tc##1{\textcolor[rgb]{0.67,0.13,1.00}{##1}}}
\expandafter\def\csname PY@tok@nb\endcsname{\def\PY@tc##1{\textcolor[rgb]{0.00,0.50,0.00}{##1}}}
\expandafter\def\csname PY@tok@nf\endcsname{\def\PY@tc##1{\textcolor[rgb]{0.00,0.00,1.00}{##1}}}
\expandafter\def\csname PY@tok@nc\endcsname{\let\PY@bf=\textbf\def\PY@tc##1{\textcolor[rgb]{0.00,0.00,1.00}{##1}}}
\expandafter\def\csname PY@tok@nn\endcsname{\let\PY@bf=\textbf\def\PY@tc##1{\textcolor[rgb]{0.00,0.00,1.00}{##1}}}
\expandafter\def\csname PY@tok@ne\endcsname{\let\PY@bf=\textbf\def\PY@tc##1{\textcolor[rgb]{0.82,0.25,0.23}{##1}}}
\expandafter\def\csname PY@tok@nv\endcsname{\def\PY@tc##1{\textcolor[rgb]{0.10,0.09,0.49}{##1}}}
\expandafter\def\csname PY@tok@no\endcsname{\def\PY@tc##1{\textcolor[rgb]{0.53,0.00,0.00}{##1}}}
\expandafter\def\csname PY@tok@nl\endcsname{\def\PY@tc##1{\textcolor[rgb]{0.63,0.63,0.00}{##1}}}
\expandafter\def\csname PY@tok@ni\endcsname{\let\PY@bf=\textbf\def\PY@tc##1{\textcolor[rgb]{0.60,0.60,0.60}{##1}}}
\expandafter\def\csname PY@tok@na\endcsname{\def\PY@tc##1{\textcolor[rgb]{0.49,0.56,0.16}{##1}}}
\expandafter\def\csname PY@tok@nt\endcsname{\let\PY@bf=\textbf\def\PY@tc##1{\textcolor[rgb]{0.00,0.50,0.00}{##1}}}
\expandafter\def\csname PY@tok@nd\endcsname{\def\PY@tc##1{\textcolor[rgb]{0.67,0.13,1.00}{##1}}}
\expandafter\def\csname PY@tok@s\endcsname{\def\PY@tc##1{\textcolor[rgb]{0.73,0.13,0.13}{##1}}}
\expandafter\def\csname PY@tok@sd\endcsname{\let\PY@it=\textit\def\PY@tc##1{\textcolor[rgb]{0.73,0.13,0.13}{##1}}}
\expandafter\def\csname PY@tok@si\endcsname{\let\PY@bf=\textbf\def\PY@tc##1{\textcolor[rgb]{0.73,0.40,0.53}{##1}}}
\expandafter\def\csname PY@tok@se\endcsname{\let\PY@bf=\textbf\def\PY@tc##1{\textcolor[rgb]{0.73,0.40,0.13}{##1}}}
\expandafter\def\csname PY@tok@sr\endcsname{\def\PY@tc##1{\textcolor[rgb]{0.73,0.40,0.53}{##1}}}
\expandafter\def\csname PY@tok@ss\endcsname{\def\PY@tc##1{\textcolor[rgb]{0.10,0.09,0.49}{##1}}}
\expandafter\def\csname PY@tok@sx\endcsname{\def\PY@tc##1{\textcolor[rgb]{0.00,0.50,0.00}{##1}}}
\expandafter\def\csname PY@tok@m\endcsname{\def\PY@tc##1{\textcolor[rgb]{0.40,0.40,0.40}{##1}}}
\expandafter\def\csname PY@tok@gh\endcsname{\let\PY@bf=\textbf\def\PY@tc##1{\textcolor[rgb]{0.00,0.00,0.50}{##1}}}
\expandafter\def\csname PY@tok@gu\endcsname{\let\PY@bf=\textbf\def\PY@tc##1{\textcolor[rgb]{0.50,0.00,0.50}{##1}}}
\expandafter\def\csname PY@tok@gd\endcsname{\def\PY@tc##1{\textcolor[rgb]{0.63,0.00,0.00}{##1}}}
\expandafter\def\csname PY@tok@gi\endcsname{\def\PY@tc##1{\textcolor[rgb]{0.00,0.63,0.00}{##1}}}
\expandafter\def\csname PY@tok@gr\endcsname{\def\PY@tc##1{\textcolor[rgb]{1.00,0.00,0.00}{##1}}}
\expandafter\def\csname PY@tok@ge\endcsname{\let\PY@it=\textit}
\expandafter\def\csname PY@tok@gs\endcsname{\let\PY@bf=\textbf}
\expandafter\def\csname PY@tok@gp\endcsname{\let\PY@bf=\textbf\def\PY@tc##1{\textcolor[rgb]{0.00,0.00,0.50}{##1}}}
\expandafter\def\csname PY@tok@go\endcsname{\def\PY@tc##1{\textcolor[rgb]{0.53,0.53,0.53}{##1}}}
\expandafter\def\csname PY@tok@gt\endcsname{\def\PY@tc##1{\textcolor[rgb]{0.00,0.27,0.87}{##1}}}
\expandafter\def\csname PY@tok@err\endcsname{\def\PY@bc##1{\setlength{\fboxsep}{0pt}\fcolorbox[rgb]{1.00,0.00,0.00}{1,1,1}{\strut ##1}}}
\expandafter\def\csname PY@tok@kc\endcsname{\let\PY@bf=\textbf\def\PY@tc##1{\textcolor[rgb]{0.00,0.50,0.00}{##1}}}
\expandafter\def\csname PY@tok@kd\endcsname{\let\PY@bf=\textbf\def\PY@tc##1{\textcolor[rgb]{0.00,0.50,0.00}{##1}}}
\expandafter\def\csname PY@tok@kn\endcsname{\let\PY@bf=\textbf\def\PY@tc##1{\textcolor[rgb]{0.00,0.50,0.00}{##1}}}
\expandafter\def\csname PY@tok@kr\endcsname{\let\PY@bf=\textbf\def\PY@tc##1{\textcolor[rgb]{0.00,0.50,0.00}{##1}}}
\expandafter\def\csname PY@tok@bp\endcsname{\def\PY@tc##1{\textcolor[rgb]{0.00,0.50,0.00}{##1}}}
\expandafter\def\csname PY@tok@fm\endcsname{\def\PY@tc##1{\textcolor[rgb]{0.00,0.00,1.00}{##1}}}
\expandafter\def\csname PY@tok@vc\endcsname{\def\PY@tc##1{\textcolor[rgb]{0.10,0.09,0.49}{##1}}}
\expandafter\def\csname PY@tok@vg\endcsname{\def\PY@tc##1{\textcolor[rgb]{0.10,0.09,0.49}{##1}}}
\expandafter\def\csname PY@tok@vi\endcsname{\def\PY@tc##1{\textcolor[rgb]{0.10,0.09,0.49}{##1}}}
\expandafter\def\csname PY@tok@vm\endcsname{\def\PY@tc##1{\textcolor[rgb]{0.10,0.09,0.49}{##1}}}
\expandafter\def\csname PY@tok@sa\endcsname{\def\PY@tc##1{\textcolor[rgb]{0.73,0.13,0.13}{##1}}}
\expandafter\def\csname PY@tok@sb\endcsname{\def\PY@tc##1{\textcolor[rgb]{0.73,0.13,0.13}{##1}}}
\expandafter\def\csname PY@tok@sc\endcsname{\def\PY@tc##1{\textcolor[rgb]{0.73,0.13,0.13}{##1}}}
\expandafter\def\csname PY@tok@dl\endcsname{\def\PY@tc##1{\textcolor[rgb]{0.73,0.13,0.13}{##1}}}
\expandafter\def\csname PY@tok@s2\endcsname{\def\PY@tc##1{\textcolor[rgb]{0.73,0.13,0.13}{##1}}}
\expandafter\def\csname PY@tok@sh\endcsname{\def\PY@tc##1{\textcolor[rgb]{0.73,0.13,0.13}{##1}}}
\expandafter\def\csname PY@tok@s1\endcsname{\def\PY@tc##1{\textcolor[rgb]{0.73,0.13,0.13}{##1}}}
\expandafter\def\csname PY@tok@mb\endcsname{\def\PY@tc##1{\textcolor[rgb]{0.40,0.40,0.40}{##1}}}
\expandafter\def\csname PY@tok@mf\endcsname{\def\PY@tc##1{\textcolor[rgb]{0.40,0.40,0.40}{##1}}}
\expandafter\def\csname PY@tok@mh\endcsname{\def\PY@tc##1{\textcolor[rgb]{0.40,0.40,0.40}{##1}}}
\expandafter\def\csname PY@tok@mi\endcsname{\def\PY@tc##1{\textcolor[rgb]{0.40,0.40,0.40}{##1}}}
\expandafter\def\csname PY@tok@il\endcsname{\def\PY@tc##1{\textcolor[rgb]{0.40,0.40,0.40}{##1}}}
\expandafter\def\csname PY@tok@mo\endcsname{\def\PY@tc##1{\textcolor[rgb]{0.40,0.40,0.40}{##1}}}
\expandafter\def\csname PY@tok@ch\endcsname{\let\PY@it=\textit\def\PY@tc##1{\textcolor[rgb]{0.25,0.50,0.50}{##1}}}
\expandafter\def\csname PY@tok@cm\endcsname{\let\PY@it=\textit\def\PY@tc##1{\textcolor[rgb]{0.25,0.50,0.50}{##1}}}
\expandafter\def\csname PY@tok@cpf\endcsname{\let\PY@it=\textit\def\PY@tc##1{\textcolor[rgb]{0.25,0.50,0.50}{##1}}}
\expandafter\def\csname PY@tok@c1\endcsname{\let\PY@it=\textit\def\PY@tc##1{\textcolor[rgb]{0.25,0.50,0.50}{##1}}}
\expandafter\def\csname PY@tok@cs\endcsname{\let\PY@it=\textit\def\PY@tc##1{\textcolor[rgb]{0.25,0.50,0.50}{##1}}}

\def\PYZbs{\char`\\}
\def\PYZus{\char`\_}
\def\PYZob{\char`\{}
\def\PYZcb{\char`\}}
\def\PYZca{\char`\^}
\def\PYZam{\char`\&}
\def\PYZlt{\char`\<}
\def\PYZgt{\char`\>}
\def\PYZsh{\char`\#}
\def\PYZpc{\char`\%}
\def\PYZdl{\char`\$}
\def\PYZhy{\char`\-}
\def\PYZsq{\char`\'}
\def\PYZdq{\char`\"}
\def\PYZti{\char`\~}
% for compatibility with earlier versions
\def\PYZat{@}
\def\PYZlb{[}
\def\PYZrb{]}
\makeatother


    % Exact colors from NB
    \definecolor{incolor}{rgb}{0.0, 0.0, 0.5}
    \definecolor{outcolor}{rgb}{0.545, 0.0, 0.0}



    
    % Prevent overflowing lines due to hard-to-break entities
    \sloppy 
    % Setup hyperref package
    \hypersetup{
      breaklinks=true,  % so long urls are correctly broken across lines
      colorlinks=true,
      urlcolor=urlcolor,
      linkcolor=linkcolor,
      citecolor=citecolor,
      }
    % Slightly bigger margins than the latex defaults
    
    \geometry{verbose,tmargin=1in,bmargin=1in,lmargin=1in,rmargin=1in}
    
    

    \begin{document}
    
    
    \maketitle
    
    

    
    \section{Machine Learning Engineer
Nanodegree}\label{machine-learning-engineer-nanodegree}

\subsection{Supervised Learning}\label{supervised-learning}

\subsection{Project 2: Building a Student Intervention
System}\label{project-2-building-a-student-intervention-system}

    Welcome to the second project of the Machine Learning Engineer
Nanodegree! In this notebook, some template code has already been
provided for you, and it will be your job to implement the additional
functionality necessary to successfully complete this project. Sections
that begin with \textbf{'Implementation'} in the header indicate that
the following block of code will require additional functionality which
you must provide. Instructions will be provided for each section and the
specifics of the implementation are marked in the code block with a
\texttt{\textquotesingle{}TODO\textquotesingle{}} statement. Please be
sure to read the instructions carefully!

In addition to implementing code, there will be questions that you must
answer which relate to the project and your implementation. Each section
where you will answer a question is preceded by a \textbf{'Question X'}
header. Carefully read each question and provide thorough answers in the
following text boxes that begin with \textbf{'Answer:'}. Your project
submission will be evaluated based on your answers to each of the
questions and the implementation you provide.

\begin{quote}
\textbf{Note:} Code and Markdown cells can be executed using the
\textbf{Shift + Enter} keyboard shortcut. In addition, Markdown cells
can be edited by typically double-clicking the cell to enter edit mode.
\end{quote}

    \subsubsection{Question 1 - Classification vs.
Regression}\label{question-1---classification-vs.-regression}

\emph{Your goal for this project is to identify students who might need
early intervention before they fail to graduate. Which type of
supervised learning problem is this, classification or regression? Why?}

    \textbf{Answer: } Classification, because the output will be binary (the
student may pass or fail to graduate). There are no underlying values
between these 2 results.

    \subsection{Exploring the Data}\label{exploring-the-data}

Run the code cell below to load necessary Python libraries and load the
student data. Note that the last column from this dataset,
\texttt{\textquotesingle{}passed\textquotesingle{}}, will be our target
label (whether the student graduated or didn't graduate). All other
columns are features about each student.

    \begin{Verbatim}[commandchars=\\\{\}]
{\color{incolor}In [{\color{incolor}114}]:} \PY{c+c1}{\PYZsh{} Import libraries}
          \PY{k+kn}{import} \PY{n+nn}{numpy} \PY{k}{as} \PY{n+nn}{np}
          \PY{k+kn}{import} \PY{n+nn}{pandas} \PY{k}{as} \PY{n+nn}{pd} 
          \PY{k+kn}{from} \PY{n+nn}{time} \PY{k}{import} \PY{n}{time}
          \PY{k+kn}{from} \PY{n+nn}{sklearn}\PY{n+nn}{.}\PY{n+nn}{metrics} \PY{k}{import} \PY{n}{f1\PYZus{}score}
          \PY{k+kn}{from} \PY{n+nn}{IPython}\PY{n+nn}{.}\PY{n+nn}{display} \PY{k}{import} \PY{n}{display}
          
          \PY{c+c1}{\PYZsh{} Read student data}
          \PY{n}{student\PYZus{}data} \PY{o}{=} \PY{n}{pd}\PY{o}{.}\PY{n}{read\PYZus{}csv}\PY{p}{(}\PY{l+s+s2}{\PYZdq{}}\PY{l+s+s2}{student\PYZhy{}data.csv}\PY{l+s+s2}{\PYZdq{}}\PY{p}{)}
          \PY{n+nb}{print} \PY{l+s+s2}{\PYZdq{}}\PY{l+s+s2}{Student data read successfully!}\PY{l+s+s2}{\PYZdq{}}
\end{Verbatim}


    \begin{Verbatim}[commandchars=\\\{\}]
Student data read successfully!

    \end{Verbatim}

    \subsubsection{Implementation: Data
Exploration}\label{implementation-data-exploration}

Let's begin by investigating the dataset to determine how many students
we have information on, and learn about the graduation rate among these
students. In the code cell below, you will need to compute the
following: - The total number of students, \texttt{n\_students}. - The
total number of features for each student, \texttt{n\_features}. - The
number of those students who passed, \texttt{n\_passed}. - The number of
those students who failed, \texttt{n\_failed}. - The graduation rate of
the class, \texttt{grad\_rate}, in percent (\%).

    \begin{Verbatim}[commandchars=\\\{\}]
{\color{incolor}In [{\color{incolor}115}]:} \PY{c+c1}{\PYZsh{} I removed my describe and counter functions from the last submission, now i\PYZsq{}m using only expressions that returns}
          \PY{c+c1}{\PYZsh{} just what the problem asks me, but you got to admit, i got it right from the first time.}
          
          \PY{c+c1}{\PYZsh{} TODO: Calculate number of students}
          \PY{n}{n\PYZus{}students} \PY{o}{=} \PY{n}{student\PYZus{}data}\PY{o}{.}\PY{n}{passed}\PY{o}{.}\PY{n}{count}\PY{p}{(}\PY{p}{)} \PY{c+c1}{\PYZsh{} i used the passed column, could be any other, ex: Pstatus will give the same value}
          
          \PY{c+c1}{\PYZsh{} TODO: Calculate number of features}
          \PY{n}{n\PYZus{}features} \PY{o}{=} \PY{n+nb}{len}\PY{p}{(}\PY{n}{student\PYZus{}data}\PY{o}{.}\PY{n}{columns}\PY{p}{)}\PY{o}{\PYZhy{}}\PY{l+m+mi}{1} \PY{c+c1}{\PYZsh{} 1 column is the passed / failed}
          
          \PY{c+c1}{\PYZsh{} TODO: Calculate passing students}
          \PY{n}{n\PYZus{}passed} \PY{o}{=} \PY{n}{student\PYZus{}data}\PY{o}{.}\PY{n}{passed}\PY{o}{.}\PY{n}{value\PYZus{}counts}\PY{p}{(}\PY{p}{)}\PY{p}{[}\PY{l+s+s1}{\PYZsq{}}\PY{l+s+s1}{yes}\PY{l+s+s1}{\PYZsq{}}\PY{p}{]}
          
          \PY{c+c1}{\PYZsh{} TODO: Calculate failing students}
          \PY{n}{n\PYZus{}failed} \PY{o}{=} \PY{n}{student\PYZus{}data}\PY{o}{.}\PY{n}{passed}\PY{o}{.}\PY{n}{value\PYZus{}counts}\PY{p}{(}\PY{p}{)}\PY{p}{[}\PY{l+s+s1}{\PYZsq{}}\PY{l+s+s1}{no}\PY{l+s+s1}{\PYZsq{}}\PY{p}{]}
          
          \PY{c+c1}{\PYZsh{} TODO: Calculate graduation rate}
          \PY{k+kn}{from} \PY{n+nn}{\PYZus{}\PYZus{}future\PYZus{}\PYZus{}} \PY{k}{import} \PY{n}{division}
          \PY{n}{grad\PYZus{}rate} \PY{o}{=} \PY{n}{n\PYZus{}passed} \PY{o}{/} \PY{n}{n\PYZus{}students} \PY{o}{*} \PY{l+m+mi}{100}
          
          \PY{c+c1}{\PYZsh{} Print the results}
          \PY{n+nb}{print} \PY{l+s+s2}{\PYZdq{}}\PY{l+s+s2}{Total number of students: }\PY{l+s+si}{\PYZob{}\PYZcb{}}\PY{l+s+s2}{\PYZdq{}}\PY{o}{.}\PY{n}{format}\PY{p}{(}\PY{n}{n\PYZus{}students}\PY{p}{)}
          \PY{n+nb}{print} \PY{l+s+s2}{\PYZdq{}}\PY{l+s+s2}{Number of features: }\PY{l+s+si}{\PYZob{}\PYZcb{}}\PY{l+s+s2}{\PYZdq{}}\PY{o}{.}\PY{n}{format}\PY{p}{(}\PY{n}{n\PYZus{}features}\PY{p}{)}
          \PY{n+nb}{print} \PY{l+s+s2}{\PYZdq{}}\PY{l+s+s2}{Number of students who passed: }\PY{l+s+si}{\PYZob{}\PYZcb{}}\PY{l+s+s2}{\PYZdq{}}\PY{o}{.}\PY{n}{format}\PY{p}{(}\PY{n}{n\PYZus{}passed}\PY{p}{)}
          \PY{n+nb}{print} \PY{l+s+s2}{\PYZdq{}}\PY{l+s+s2}{Graduation rate of the class: }\PY{l+s+si}{\PYZob{}:.2f\PYZcb{}}\PY{l+s+s2}{\PYZpc{}}\PY{l+s+s2}{\PYZdq{}}\PY{o}{.}\PY{n}{format}\PY{p}{(}\PY{n}{grad\PYZus{}rate}\PY{p}{)}
\end{Verbatim}


    \begin{Verbatim}[commandchars=\\\{\}]
Total number of students: 395
Number of features: 30
Number of students who passed: 265
Graduation rate of the class: 67.09\%

    \end{Verbatim}

    \subsection{Preparing the Data}\label{preparing-the-data}

In this section, we will prepare the data for modeling, training and
testing.

\subsubsection{Identify feature and target
columns}\label{identify-feature-and-target-columns}

It is often the case that the data you obtain contains non-numeric
features. This can be a problem, as most machine learning algorithms
expect numeric data to perform computations with.

Run the code cell below to separate the student data into feature and
target columns to see if any features are non-numeric.

    \begin{Verbatim}[commandchars=\\\{\}]
{\color{incolor}In [{\color{incolor}116}]:} \PY{c+c1}{\PYZsh{} Extract feature columns}
          \PY{n}{feature\PYZus{}cols} \PY{o}{=} \PY{n+nb}{list}\PY{p}{(}\PY{n}{student\PYZus{}data}\PY{o}{.}\PY{n}{columns}\PY{p}{[}\PY{p}{:}\PY{o}{\PYZhy{}}\PY{l+m+mi}{1}\PY{p}{]}\PY{p}{)}
          
          \PY{c+c1}{\PYZsh{} Extract target column \PYZsq{}passed\PYZsq{}}
          \PY{n}{target\PYZus{}col} \PY{o}{=} \PY{n}{student\PYZus{}data}\PY{o}{.}\PY{n}{columns}\PY{p}{[}\PY{o}{\PYZhy{}}\PY{l+m+mi}{1}\PY{p}{]} 
          
          \PY{c+c1}{\PYZsh{} Show the list of columns}
          \PY{n+nb}{print} \PY{l+s+s2}{\PYZdq{}}\PY{l+s+s2}{Feature columns:}\PY{l+s+se}{\PYZbs{}n}\PY{l+s+si}{\PYZob{}\PYZcb{}}\PY{l+s+s2}{\PYZdq{}}\PY{o}{.}\PY{n}{format}\PY{p}{(}\PY{n}{feature\PYZus{}cols}\PY{p}{)}
          \PY{n+nb}{print} \PY{l+s+s2}{\PYZdq{}}\PY{l+s+se}{\PYZbs{}n}\PY{l+s+s2}{Target column: }\PY{l+s+si}{\PYZob{}\PYZcb{}}\PY{l+s+s2}{\PYZdq{}}\PY{o}{.}\PY{n}{format}\PY{p}{(}\PY{n}{target\PYZus{}col}\PY{p}{)}
          
          \PY{c+c1}{\PYZsh{} Separate the data into feature data and target data (X\PYZus{}all and y\PYZus{}all, respectively)}
          \PY{n}{X\PYZus{}all} \PY{o}{=} \PY{n}{student\PYZus{}data}\PY{p}{[}\PY{n}{feature\PYZus{}cols}\PY{p}{]}
          \PY{n}{y\PYZus{}all} \PY{o}{=} \PY{n}{student\PYZus{}data}\PY{p}{[}\PY{n}{target\PYZus{}col}\PY{p}{]}
          
          \PY{c+c1}{\PYZsh{} Show the feature information by printing the first five rows}
          \PY{n+nb}{print} \PY{l+s+s2}{\PYZdq{}}\PY{l+s+se}{\PYZbs{}n}\PY{l+s+s2}{Feature values:}\PY{l+s+s2}{\PYZdq{}}
          \PY{n+nb}{print} \PY{n}{X\PYZus{}all}\PY{o}{.}\PY{n}{head}\PY{p}{(}\PY{p}{)}
\end{Verbatim}


    \begin{Verbatim}[commandchars=\\\{\}]
Feature columns:
['school', 'sex', 'age', 'address', 'famsize', 'Pstatus', 'Medu', 'Fedu', 'Mjob', 'Fjob', 'reason', 'guardian', 'traveltime', 'studytime', 'failures', 'schoolsup', 'famsup', 'paid', 'activities', 'nursery', 'higher', 'internet', 'romantic', 'famrel', 'freetime', 'goout', 'Dalc', 'Walc', 'health', 'absences']

Target column: passed

Feature values:
  school sex  age address famsize Pstatus  Medu  Fedu     Mjob      Fjob  \textbackslash{}
0     GP   F   18       U     GT3       A     4     4  at\_home   teacher   
1     GP   F   17       U     GT3       T     1     1  at\_home     other   
2     GP   F   15       U     LE3       T     1     1  at\_home     other   
3     GP   F   15       U     GT3       T     4     2   health  services   
4     GP   F   16       U     GT3       T     3     3    other     other   

    {\ldots}    higher internet  romantic  famrel  freetime goout Dalc Walc health  \textbackslash{}
0   {\ldots}       yes       no        no       4         3     4    1    1      3   
1   {\ldots}       yes      yes        no       5         3     3    1    1      3   
2   {\ldots}       yes      yes        no       4         3     2    2    3      3   
3   {\ldots}       yes      yes       yes       3         2     2    1    1      5   
4   {\ldots}       yes       no        no       4         3     2    1    2      5   

  absences  
0        6  
1        4  
2       10  
3        2  
4        4  

[5 rows x 30 columns]

    \end{Verbatim}

    \subsubsection{Preprocess Feature
Columns}\label{preprocess-feature-columns}

As you can see, there are several non-numeric columns that need to be
converted! Many of them are simply \texttt{yes}/\texttt{no}, e.g.
\texttt{internet}. These can be reasonably converted into
\texttt{1}/\texttt{0} (binary) values.

Other columns, like \texttt{Mjob} and \texttt{Fjob}, have more than two
values, and are known as \emph{categorical variables}. The recommended
way to handle such a column is to create as many columns as possible
values (e.g. \texttt{Fjob\_teacher}, \texttt{Fjob\_other},
\texttt{Fjob\_services}, etc.), and assign a \texttt{1} to one of them
and \texttt{0} to all others.

These generated columns are sometimes called \emph{dummy variables}, and
we will use the
\href{http://pandas.pydata.org/pandas-docs/stable/generated/pandas.get_dummies.html?highlight=get_dummies\#pandas.get_dummies}{\texttt{pandas.get\_dummies()}}
function to perform this transformation. Run the code cell below to
perform the preprocessing routine discussed in this section.

    \begin{Verbatim}[commandchars=\\\{\}]
{\color{incolor}In [{\color{incolor}117}]:} \PY{k}{def} \PY{n+nf}{preprocess\PYZus{}features}\PY{p}{(}\PY{n}{X}\PY{p}{)}\PY{p}{:}
              \PY{l+s+sd}{\PYZsq{}\PYZsq{}\PYZsq{} Preprocesses the student data and converts non\PYZhy{}numeric binary variables into}
          \PY{l+s+sd}{        binary (0/1) variables. Converts categorical variables into dummy variables. \PYZsq{}\PYZsq{}\PYZsq{}}
              
              \PY{c+c1}{\PYZsh{} Initialize new output DataFrame}
              \PY{n}{output} \PY{o}{=} \PY{n}{pd}\PY{o}{.}\PY{n}{DataFrame}\PY{p}{(}\PY{n}{index} \PY{o}{=} \PY{n}{X}\PY{o}{.}\PY{n}{index}\PY{p}{)}
          
              \PY{c+c1}{\PYZsh{} Investigate each feature column for the data}
              \PY{k}{for} \PY{n}{col}\PY{p}{,} \PY{n}{col\PYZus{}data} \PY{o+ow}{in} \PY{n}{X}\PY{o}{.}\PY{n}{iteritems}\PY{p}{(}\PY{p}{)}\PY{p}{:}
                  
                  \PY{c+c1}{\PYZsh{} If data type is non\PYZhy{}numeric, replace all yes/no values with 1/0}
                  \PY{k}{if} \PY{n}{col\PYZus{}data}\PY{o}{.}\PY{n}{dtype} \PY{o}{==} \PY{n+nb}{object}\PY{p}{:}
                      \PY{n}{col\PYZus{}data} \PY{o}{=} \PY{n}{col\PYZus{}data}\PY{o}{.}\PY{n}{replace}\PY{p}{(}\PY{p}{[}\PY{l+s+s1}{\PYZsq{}}\PY{l+s+s1}{yes}\PY{l+s+s1}{\PYZsq{}}\PY{p}{,} \PY{l+s+s1}{\PYZsq{}}\PY{l+s+s1}{no}\PY{l+s+s1}{\PYZsq{}}\PY{p}{]}\PY{p}{,} \PY{p}{[}\PY{l+m+mi}{1}\PY{p}{,} \PY{l+m+mi}{0}\PY{p}{]}\PY{p}{)}
          
                  \PY{c+c1}{\PYZsh{} If data type is categorical, convert to dummy variables}
                  \PY{k}{if} \PY{n}{col\PYZus{}data}\PY{o}{.}\PY{n}{dtype} \PY{o}{==} \PY{n+nb}{object}\PY{p}{:}
                      \PY{c+c1}{\PYZsh{} Example: \PYZsq{}school\PYZsq{} =\PYZgt{} \PYZsq{}school\PYZus{}GP\PYZsq{} and \PYZsq{}school\PYZus{}MS\PYZsq{}}
                      \PY{n}{col\PYZus{}data} \PY{o}{=} \PY{n}{pd}\PY{o}{.}\PY{n}{get\PYZus{}dummies}\PY{p}{(}\PY{n}{col\PYZus{}data}\PY{p}{,} \PY{n}{prefix} \PY{o}{=} \PY{n}{col}\PY{p}{)}  
                  
                  \PY{c+c1}{\PYZsh{} Collect the revised columns}
                  \PY{n}{output} \PY{o}{=} \PY{n}{output}\PY{o}{.}\PY{n}{join}\PY{p}{(}\PY{n}{col\PYZus{}data}\PY{p}{)}
              
              \PY{k}{return} \PY{n}{output}
          
          \PY{n}{X\PYZus{}all} \PY{o}{=} \PY{n}{preprocess\PYZus{}features}\PY{p}{(}\PY{n}{X\PYZus{}all}\PY{p}{)}
          \PY{n+nb}{print} \PY{l+s+s2}{\PYZdq{}}\PY{l+s+s2}{Processed feature columns (}\PY{l+s+si}{\PYZob{}\PYZcb{}}\PY{l+s+s2}{ total features):}\PY{l+s+se}{\PYZbs{}n}\PY{l+s+si}{\PYZob{}\PYZcb{}}\PY{l+s+s2}{\PYZdq{}}\PY{o}{.}\PY{n}{format}\PY{p}{(}\PY{n+nb}{len}\PY{p}{(}\PY{n}{X\PYZus{}all}\PY{o}{.}\PY{n}{columns}\PY{p}{)}\PY{p}{,} \PY{n+nb}{list}\PY{p}{(}\PY{n}{X\PYZus{}all}\PY{o}{.}\PY{n}{columns}\PY{p}{)}\PY{p}{)}
\end{Verbatim}


    \begin{Verbatim}[commandchars=\\\{\}]
Processed feature columns (48 total features):
['school\_GP', 'school\_MS', 'sex\_F', 'sex\_M', 'age', 'address\_R', 'address\_U', 'famsize\_GT3', 'famsize\_LE3', 'Pstatus\_A', 'Pstatus\_T', 'Medu', 'Fedu', 'Mjob\_at\_home', 'Mjob\_health', 'Mjob\_other', 'Mjob\_services', 'Mjob\_teacher', 'Fjob\_at\_home', 'Fjob\_health', 'Fjob\_other', 'Fjob\_services', 'Fjob\_teacher', 'reason\_course', 'reason\_home', 'reason\_other', 'reason\_reputation', 'guardian\_father', 'guardian\_mother', 'guardian\_other', 'traveltime', 'studytime', 'failures', 'schoolsup', 'famsup', 'paid', 'activities', 'nursery', 'higher', 'internet', 'romantic', 'famrel', 'freetime', 'goout', 'Dalc', 'Walc', 'health', 'absences']

    \end{Verbatim}

    \subsubsection{Implementation: Training and Testing Data
Split}\label{implementation-training-and-testing-data-split}

So far, we have converted all \emph{categorical} features into numeric
values. For the next step, we split the data (both features and
corresponding labels) into training and test sets. In the following code
cell below, you will need to implement the following: - Randomly shuffle
and split the data (\texttt{X\_all}, \texttt{y\_all}) into training and
testing subsets. - Use 300 training points (approximately 75\%) and 95
testing points (approximately 25\%). - Set a \texttt{random\_state} for
the function(s) you use, if provided. - Store the results in
\texttt{X\_train}, \texttt{X\_test}, \texttt{y\_train}, and
\texttt{y\_test}.

    \begin{Verbatim}[commandchars=\\\{\}]
{\color{incolor}In [{\color{incolor}118}]:} \PY{c+c1}{\PYZsh{} TODO: Import any additional functionality you may need here}
          \PY{k+kn}{from} \PY{n+nn}{sklearn}\PY{n+nn}{.}\PY{n+nn}{cross\PYZus{}validation} \PY{k}{import} \PY{n}{train\PYZus{}test\PYZus{}split}
          
          \PY{c+c1}{\PYZsh{} TODO: Set the number of training points}
          \PY{n}{num\PYZus{}train} \PY{o}{=} \PY{l+m+mi}{300}
          
          \PY{c+c1}{\PYZsh{} Set the number of testing points}
          \PY{n}{num\PYZus{}test} \PY{o}{=} \PY{n}{X\PYZus{}all}\PY{o}{.}\PY{n}{shape}\PY{p}{[}\PY{l+m+mi}{0}\PY{p}{]} \PY{o}{\PYZhy{}} \PY{n}{num\PYZus{}train}
          
          \PY{c+c1}{\PYZsh{} TODO: Shuffle and split the dataset into the number of training and testing points above}
          \PY{n}{X\PYZus{}train}\PY{p}{,} \PY{n}{X\PYZus{}test}\PY{p}{,} \PY{n}{y\PYZus{}train}\PY{p}{,} \PY{n}{y\PYZus{}test}  \PY{o}{=} \PY{n}{train\PYZus{}test\PYZus{}split}\PY{p}{(}\PY{n}{X\PYZus{}all}\PY{p}{,} \PY{n}{y\PYZus{}all}\PY{p}{,} \PY{n}{train\PYZus{}size}\PY{o}{=}\PY{n}{num\PYZus{}train}\PY{p}{,} \PY{n}{stratify}\PY{o}{=}\PY{n}{y\PYZus{}all}\PY{p}{,} \PY{n}{test\PYZus{}size}\PY{o}{=}\PY{l+m+mf}{0.24}\PY{p}{,} \PY{n}{random\PYZus{}state}\PY{o}{=}\PY{l+m+mi}{42}\PY{p}{)}
          \PY{c+c1}{\PYZsh{}X\PYZus{}train, X\PYZus{}test, y\PYZus{}train, y\PYZus{}test = train\PYZus{}test\PYZus{}split(X\PYZus{}all, y\PYZus{}all, train\PYZus{}size=num\PYZus{}train, test\PYZus{}size=num\PYZus{}test, random\PYZus{}state=42)}
          
          \PY{c+c1}{\PYZsh{} Show the results of the split}
          \PY{n+nb}{print} \PY{l+s+s2}{\PYZdq{}}\PY{l+s+s2}{Training set has }\PY{l+s+si}{\PYZob{}\PYZcb{}}\PY{l+s+s2}{ samples.}\PY{l+s+s2}{\PYZdq{}}\PY{o}{.}\PY{n}{format}\PY{p}{(}\PY{n}{X\PYZus{}train}\PY{o}{.}\PY{n}{shape}\PY{p}{[}\PY{l+m+mi}{0}\PY{p}{]}\PY{p}{)}
          \PY{n+nb}{print} \PY{l+s+s2}{\PYZdq{}}\PY{l+s+s2}{Testing set has }\PY{l+s+si}{\PYZob{}\PYZcb{}}\PY{l+s+s2}{ samples.}\PY{l+s+s2}{\PYZdq{}}\PY{o}{.}\PY{n}{format}\PY{p}{(}\PY{n}{X\PYZus{}test}\PY{o}{.}\PY{n}{shape}\PY{p}{[}\PY{l+m+mi}{0}\PY{p}{]}\PY{p}{)}
\end{Verbatim}


    \begin{Verbatim}[commandchars=\\\{\}]
Training set has 300 samples.
Testing set has 95 samples.

    \end{Verbatim}

    \subsection{Training and Evaluating
Models}\label{training-and-evaluating-models}

In this section, you will choose 3 supervised learning models that are
appropriate for this problem and available in \texttt{scikit-learn}. You
will first discuss the reasoning behind choosing these three models by
considering what you know about the data and each model's strengths and
weaknesses. You will then fit the model to varying sizes of training
data (100 data points, 200 data points, and 300 data points) and measure
the F1 score. You will need to produce three tables (one for each model)
that shows the training set size, training time, prediction time, F1
score on the training set, and F1 score on the testing set.

\textbf{The following supervised learning models are currently available
in}
\href{http://scikit-learn.org/stable/supervised_learning.html}{\texttt{scikit-learn}}
\textbf{that you may choose from:} - Gaussian Naive Bayes (GaussianNB) -
Decision Trees - Ensemble Methods (Bagging, AdaBoost, Random Forest,
Gradient Boosting) - K-Nearest Neighbors (KNeighbors) - Stochastic
Gradient Descent (SGDC) - Support Vector Machines (SVM) - Logistic
Regression

    \subsubsection{Question 2 - Model
Application}\label{question-2---model-application}

\emph{List three supervised learning models that are appropriate for
this problem. For each model chosen} - Describe one real-world
application in industry where the model can be applied. \emph{(You may
need to do a small bit of research for this --- give references!)} -
What are the strengths of the model; when does it perform well? - What
are the weaknesses of the model; when does it perform poorly? - What
makes this model a good candidate for the problem, given what you know
about the data?

    \textbf{Answer: } Naive Bayes, SVM and Random Forest were chosen.

\subsection{Naive Bayes}\label{naive-bayes}

\paragraph{Naive Bayes applicability:}\label{naive-bayes-applicability}

The emails providers that we use today utilize this method to classify
emails as spams and no spams. This model was selected because it
utilizes a naive classifier, it utilizes the Bayes rule to get a subset
of words from the mail and uses a bag of words previously trainned in
the model to take its decision, this will serve as base for the spam /
no spam guess as you can see in this article here:

https://web.stanford.edu/class/cs124/lec/naivebayes.pdf

\paragraph{Naive Bayes strenghts:}\label{naive-bayes-strenghts}

One of the fastest models available;

It takes features independently. In the end, all features contribute to
the probability of the output (Naive);

Performs better in categorical values;

\paragraph{Naive Bayes Weaknesses:}\label{naive-bayes-weaknesses}

It doesn't consider records from test data which weren't present in the
training data. Proper data treatment could reduce the amount of
disconsidered values.

Does not perform well in numerical values;

\paragraph{Why it's a good model for this
problem:}\label{why-its-a-good-model-for-this-problem}

Most of our data is categorical, so we can expect a good performance
with a low computational costs.

\subsection{Support Vector Machines}\label{support-vector-machines}

\paragraph{Support Vector Machines
applicability:}\label{support-vector-machines-applicability}

In chemistry, elements tend to have particular characteristics that are
integer, such as atomic mass, charge, eletronic spin, etc, the SVM
method takes advantage of those clear margins between the features to
classify the records utilizing lines, planes and hyperplanes (depending
on the record dimensionality) as we can see in this article (page 59),
there's a section where they use this algorithm to classify the
similarity between sequences of aminoacidis, this is a key point to
predict this sequence biological properties.

http://www.ivanciuc.org/Files/Reprint/Ivanciuc\_Applications\_of\_Support\_Vector\_Machines\_in\_Chemistry.pdf

\paragraph{Support Vector Machines
strenghts:}\label{support-vector-machines-strenghts}

It works better when there's a clear margin between the classifiers;

It works well in models with several dimensions;

It is effective when there are more dimensions than samples;

\paragraph{Support Vector Machines
weaknesses:}\label{support-vector-machines-weaknesses}

It requires a lot of processing to perform;

It doesn't perform well when dimensions are overlaping;

\paragraph{Why it's a good model for this
problem:}\label{why-its-a-good-model-for-this-problem-1}

Dataset is small, so there is no concern about performance. Therefore,
this model would fit if the classes are free from too much noise.

\subsection{Random Forest}\label{random-forest}

\paragraph{Random Forest
applicability:}\label{random-forest-applicability}

Random Forest is the method that creates several classifiers and
aggregate its results, this method can be implemented in computer vision
to classify parts of the body, it basically generates several different
decision trees and once new data comes, the algorithm passes this data
in each decision tree, they all take its guess about what part this data
is and than we see which part got the most votes. We can see an
application of this method in this article:

http://pages.iai.uni-bonn.de/frintrop\_simone//BVW13/BVW-gall.pdf

\paragraph{Random Forest strenghts:}\label{random-forest-strenghts}

It has a great performance among other algorithms;

It runs well enough in large datasets;

The overfitting can be reduced by increasing the number of trees;

\paragraph{Random Forest weaknesses:}\label{random-forest-weaknesses}

It can be very sensitive to small perturbations on the data;

It's hard to avoid overfitting. Prior analysis are required for that;

\paragraph{Why it's a good model for this
problem:}\label{why-its-a-good-model-for-this-problem-2}

This model can generalize very well among other models and it shows a
very high rate of accuracy.

    \subsubsection{Setup}\label{setup}

Run the code cell below to initialize three helper functions which you
can use for training and testing the three supervised learning models
you've chosen above. The functions are as follows: -
\texttt{train\_classifier} - takes as input a classifier and training
data and fits the classifier to the data. - \texttt{predict\_labels} -
takes as input a fit classifier, features, and a target labeling and
makes predictions using the F1 score. - \texttt{train\_predict} - takes
as input a classifier, and the training and testing data, and performs
\texttt{train\_clasifier} and \texttt{predict\_labels}. - This function
will report the F1 score for both the training and testing data
separately.

    \begin{Verbatim}[commandchars=\\\{\}]
{\color{incolor}In [{\color{incolor}119}]:} \PY{k}{def} \PY{n+nf}{train\PYZus{}classifier}\PY{p}{(}\PY{n}{clf}\PY{p}{,} \PY{n}{X\PYZus{}train}\PY{p}{,} \PY{n}{y\PYZus{}train}\PY{p}{)}\PY{p}{:}
              \PY{l+s+sd}{\PYZsq{}\PYZsq{}\PYZsq{} Fits a classifier to the training data. \PYZsq{}\PYZsq{}\PYZsq{}}
              
              \PY{c+c1}{\PYZsh{} Start the clock, train the classifier, then stop the clock}
              \PY{n}{start} \PY{o}{=} \PY{n}{time}\PY{p}{(}\PY{p}{)}
              \PY{n}{clf}\PY{o}{.}\PY{n}{fit}\PY{p}{(}\PY{n}{X\PYZus{}train}\PY{p}{,} \PY{n}{y\PYZus{}train}\PY{p}{)}
              \PY{n}{end} \PY{o}{=} \PY{n}{time}\PY{p}{(}\PY{p}{)}
              
              \PY{c+c1}{\PYZsh{} Print the results}
              \PY{n+nb}{print} \PY{l+s+s2}{\PYZdq{}}\PY{l+s+s2}{Trained model in }\PY{l+s+si}{\PYZob{}:.4f\PYZcb{}}\PY{l+s+s2}{ seconds}\PY{l+s+s2}{\PYZdq{}}\PY{o}{.}\PY{n}{format}\PY{p}{(}\PY{n}{end} \PY{o}{\PYZhy{}} \PY{n}{start}\PY{p}{)}
          
              
          \PY{k}{def} \PY{n+nf}{predict\PYZus{}labels}\PY{p}{(}\PY{n}{clf}\PY{p}{,} \PY{n}{features}\PY{p}{,} \PY{n}{target}\PY{p}{)}\PY{p}{:}
              \PY{l+s+sd}{\PYZsq{}\PYZsq{}\PYZsq{} Makes predictions using a fit classifier based on F1 score. \PYZsq{}\PYZsq{}\PYZsq{}}
              
              \PY{c+c1}{\PYZsh{} Start the clock, make predictions, then stop the clock}
              \PY{n}{start} \PY{o}{=} \PY{n}{time}\PY{p}{(}\PY{p}{)}
              \PY{n}{y\PYZus{}pred} \PY{o}{=} \PY{n}{clf}\PY{o}{.}\PY{n}{predict}\PY{p}{(}\PY{n}{features}\PY{p}{)}
              \PY{n}{end} \PY{o}{=} \PY{n}{time}\PY{p}{(}\PY{p}{)}
              
              \PY{c+c1}{\PYZsh{} Print and return results}
              \PY{n+nb}{print} \PY{l+s+s2}{\PYZdq{}}\PY{l+s+s2}{Made predictions in }\PY{l+s+si}{\PYZob{}:.4f\PYZcb{}}\PY{l+s+s2}{ seconds.}\PY{l+s+s2}{\PYZdq{}}\PY{o}{.}\PY{n}{format}\PY{p}{(}\PY{n}{end} \PY{o}{\PYZhy{}} \PY{n}{start}\PY{p}{)}
              \PY{k}{return} \PY{n}{f1\PYZus{}score}\PY{p}{(}\PY{n}{target}\PY{o}{.}\PY{n}{values}\PY{p}{,} \PY{n}{y\PYZus{}pred}\PY{p}{,} \PY{n}{pos\PYZus{}label}\PY{o}{=}\PY{l+s+s1}{\PYZsq{}}\PY{l+s+s1}{yes}\PY{l+s+s1}{\PYZsq{}}\PY{p}{)}
          
          
          \PY{k}{def} \PY{n+nf}{train\PYZus{}predict}\PY{p}{(}\PY{n}{clf}\PY{p}{,} \PY{n}{X\PYZus{}train}\PY{p}{,} \PY{n}{y\PYZus{}train}\PY{p}{,} \PY{n}{X\PYZus{}test}\PY{p}{,} \PY{n}{y\PYZus{}test}\PY{p}{)}\PY{p}{:}
              \PY{l+s+sd}{\PYZsq{}\PYZsq{}\PYZsq{} Train and predict using a classifer based on F1 score. \PYZsq{}\PYZsq{}\PYZsq{}}
              
              \PY{c+c1}{\PYZsh{} Indicate the classifier and the training set size}
              \PY{n+nb}{print} \PY{l+s+s2}{\PYZdq{}}\PY{l+s+s2}{Training a }\PY{l+s+si}{\PYZob{}\PYZcb{}}\PY{l+s+s2}{ using a training set size of }\PY{l+s+si}{\PYZob{}\PYZcb{}}\PY{l+s+s2}{. . .}\PY{l+s+s2}{\PYZdq{}}\PY{o}{.}\PY{n}{format}\PY{p}{(}\PY{n}{clf}\PY{o}{.}\PY{n+nv+vm}{\PYZus{}\PYZus{}class\PYZus{}\PYZus{}}\PY{o}{.}\PY{n+nv+vm}{\PYZus{}\PYZus{}name\PYZus{}\PYZus{}}\PY{p}{,} \PY{n+nb}{len}\PY{p}{(}\PY{n}{X\PYZus{}train}\PY{p}{)}\PY{p}{)}
              
              \PY{c+c1}{\PYZsh{} Train the classifier}
              \PY{n}{train\PYZus{}classifier}\PY{p}{(}\PY{n}{clf}\PY{p}{,} \PY{n}{X\PYZus{}train}\PY{p}{,} \PY{n}{y\PYZus{}train}\PY{p}{)}
              
              \PY{c+c1}{\PYZsh{} Print the results of prediction for both training and testing}
              \PY{n+nb}{print} \PY{l+s+s2}{\PYZdq{}}\PY{l+s+s2}{F1 score for training set: }\PY{l+s+si}{\PYZob{}:.4f\PYZcb{}}\PY{l+s+s2}{.}\PY{l+s+s2}{\PYZdq{}}\PY{o}{.}\PY{n}{format}\PY{p}{(}\PY{n}{predict\PYZus{}labels}\PY{p}{(}\PY{n}{clf}\PY{p}{,} \PY{n}{X\PYZus{}train}\PY{p}{,} \PY{n}{y\PYZus{}train}\PY{p}{)}\PY{p}{)}
              \PY{n+nb}{print} \PY{l+s+s2}{\PYZdq{}}\PY{l+s+s2}{F1 score for test set: }\PY{l+s+si}{\PYZob{}:.4f\PYZcb{}}\PY{l+s+s2}{.}\PY{l+s+s2}{\PYZdq{}}\PY{o}{.}\PY{n}{format}\PY{p}{(}\PY{n}{predict\PYZus{}labels}\PY{p}{(}\PY{n}{clf}\PY{p}{,} \PY{n}{X\PYZus{}test}\PY{p}{,} \PY{n}{y\PYZus{}test}\PY{p}{)}\PY{p}{)}
\end{Verbatim}


    \subsubsection{Implementation: Model Performance
Metrics}\label{implementation-model-performance-metrics}

With the predefined functions above, you will now import the three
supervised learning models of your choice and run the
\texttt{train\_predict} function for each one. Remember that you will
need to train and predict on each classifier for three different
training set sizes: 100, 200, and 300. Hence, you should expect to have
9 different outputs below --- 3 for each model using the varying
training set sizes. In the following code cell, you will need to
implement the following: - Import the three supervised learning models
you've discussed in the previous section. - Initialize the three models
and store them in \texttt{clf\_A}, \texttt{clf\_B}, and \texttt{clf\_C}.
- Use a \texttt{random\_state} for each model you use, if provided. -
\textbf{Note:} Use the default settings for each model --- you will tune
one specific model in a later section. - Create the different training
set sizes to be used to train each model. - \emph{Do not reshuffle and
resplit the data! The new training points should be drawn from
\texttt{X\_train} and \texttt{y\_train}.} - Fit each model with each
training set size and make predictions on the test set (9 in total).\\
\textbf{Note:} Three tables are provided after the following code cell
which can be used to store your results.

    \begin{Verbatim}[commandchars=\\\{\}]
{\color{incolor}In [{\color{incolor}120}]:} \PY{c+c1}{\PYZsh{} TODO: Import the three supervised learning models from sklearn}
          \PY{k+kn}{from} \PY{n+nn}{sklearn}\PY{n+nn}{.}\PY{n+nn}{naive\PYZus{}bayes} \PY{k}{import} \PY{n}{GaussianNB}
          \PY{k+kn}{from} \PY{n+nn}{sklearn}\PY{n+nn}{.}\PY{n+nn}{svm} \PY{k}{import} \PY{n}{SVC}
          \PY{k+kn}{from} \PY{n+nn}{sklearn}\PY{n+nn}{.}\PY{n+nn}{ensemble} \PY{k}{import} \PY{n}{RandomForestClassifier}
          
          \PY{c+c1}{\PYZsh{} TODO: Initialize the three models}
          \PY{n}{clf\PYZus{}A} \PY{o}{=} \PY{n}{GaussianNB}\PY{p}{(}\PY{p}{)}
          \PY{n}{clf\PYZus{}B} \PY{o}{=} \PY{n}{SVC}\PY{p}{(}\PY{n}{random\PYZus{}state}\PY{o}{=}\PY{l+m+mi}{42}\PY{p}{)}
          \PY{n}{clf\PYZus{}C} \PY{o}{=} \PY{n}{RandomForestClassifier}\PY{p}{(}\PY{n}{random\PYZus{}state}\PY{o}{=}\PY{l+m+mi}{42}\PY{p}{)}
          
          \PY{c+c1}{\PYZsh{} TODO: Set up the training set sizes}
          \PY{k}{for} \PY{n}{clf} \PY{o+ow}{in} \PY{p}{[}\PY{n}{clf\PYZus{}A}\PY{p}{,} \PY{n}{clf\PYZus{}B}\PY{p}{,} \PY{n}{clf\PYZus{}C}\PY{p}{]}\PY{p}{:}
              \PY{k}{for} \PY{n}{n} \PY{o+ow}{in} \PY{p}{[}\PY{l+m+mi}{300}\PY{p}{,} \PY{l+m+mi}{200}\PY{p}{,} \PY{l+m+mi}{100}\PY{p}{]}\PY{p}{:}
                  \PY{n}{X\PYZus{}train\PYZus{}n} \PY{o}{=} \PY{n}{X\PYZus{}train}\PY{p}{[}\PY{p}{:}\PY{n}{n}\PY{p}{]}
                  \PY{n}{y\PYZus{}train\PYZus{}n} \PY{o}{=} \PY{n}{y\PYZus{}train}\PY{p}{[}\PY{p}{:}\PY{n}{n}\PY{p}{]}
                  \PY{n}{train\PYZus{}predict}\PY{p}{(}\PY{n}{clf}\PY{p}{,} \PY{n}{X\PYZus{}train\PYZus{}n}\PY{p}{,} \PY{n}{y\PYZus{}train\PYZus{}n}\PY{p}{,} \PY{n}{X\PYZus{}test}\PY{p}{,} \PY{n}{y\PYZus{}test}\PY{p}{)}
                  \PY{n+nb}{print} \PY{l+s+s1}{\PYZsq{}}\PY{l+s+se}{\PYZbs{}n}\PY{l+s+s1}{\PYZsq{}}
\end{Verbatim}


    \begin{Verbatim}[commandchars=\\\{\}]
Training a GaussianNB using a training set size of 300. . .
Trained model in 0.0020 seconds
Made predictions in 0.0010 seconds.
F1 score for training set: 0.8134.
Made predictions in 0.0010 seconds.
F1 score for test set: 0.7761.


Training a GaussianNB using a training set size of 200. . .
Trained model in 0.0020 seconds
Made predictions in 0.0010 seconds.
F1 score for training set: 0.8060.
Made predictions in 0.0000 seconds.
F1 score for test set: 0.7218.


Training a GaussianNB using a training set size of 100. . .
Trained model in 0.0010 seconds
Made predictions in 0.0000 seconds.
F1 score for training set: 0.7752.
Made predictions in 0.0010 seconds.
F1 score for test set: 0.6457.


Training a SVC using a training set size of 300. . .
Trained model in 0.0160 seconds
Made predictions in 0.0090 seconds.
F1 score for training set: 0.8664.
Made predictions in 0.0030 seconds.
F1 score for test set: 0.8052.


Training a SVC using a training set size of 200. . .
Trained model in 0.0080 seconds
Made predictions in 0.0040 seconds.
F1 score for training set: 0.8431.
Made predictions in 0.0020 seconds.
F1 score for test set: 0.8105.


Training a SVC using a training set size of 100. . .
Trained model in 0.0020 seconds
Made predictions in 0.0020 seconds.
F1 score for training set: 0.8354.
Made predictions in 0.0010 seconds.
F1 score for test set: 0.8025.


Training a RandomForestClassifier using a training set size of 300. . .
Trained model in 0.0320 seconds
Made predictions in 0.0020 seconds.
F1 score for training set: 0.9975.
Made predictions in 0.0020 seconds.
F1 score for test set: 0.7132.


Training a RandomForestClassifier using a training set size of 200. . .
Trained model in 0.0270 seconds
Made predictions in 0.0020 seconds.
F1 score for training set: 0.9885.
Made predictions in 0.0010 seconds.
F1 score for test set: 0.6822.


Training a RandomForestClassifier using a training set size of 100. . .
Trained model in 0.0250 seconds
Made predictions in 0.0020 seconds.
F1 score for training set: 0.9924.
Made predictions in 0.0010 seconds.
F1 score for test set: 0.7368.



    \end{Verbatim}

    \subsubsection{Tabular Results}\label{tabular-results}

Edit the cell below to see how a table can be designed in
\href{https://github.com/adam-p/markdown-here/wiki/Markdown-Cheatsheet\#tables}{Markdown}.
You can record your results from above in the tables provided.

    ** Classifer 1 - Naive Bayes**

\begin{longtable}[]{@{}ccccc@{}}
\toprule
\begin{minipage}[b]{0.16\columnwidth}\centering\strut
Training Set Size\strut
\end{minipage} & \begin{minipage}[b]{0.21\columnwidth}\centering\strut
Training Time\strut
\end{minipage} & \begin{minipage}[b]{0.20\columnwidth}\centering\strut
Prediction Time (test)\strut
\end{minipage} & \begin{minipage}[b]{0.15\columnwidth}\centering\strut
F1 Score (train)\strut
\end{minipage} & \begin{minipage}[b]{0.14\columnwidth}\centering\strut
F1 Score (test)\strut
\end{minipage}\tabularnewline
\midrule
\endhead
\begin{minipage}[t]{0.16\columnwidth}\centering\strut
100\strut
\end{minipage} & \begin{minipage}[t]{0.21\columnwidth}\centering\strut
0.0010\strut
\end{minipage} & \begin{minipage}[t]{0.20\columnwidth}\centering\strut
0.0000\strut
\end{minipage} & \begin{minipage}[t]{0.15\columnwidth}\centering\strut
0.7752\strut
\end{minipage} & \begin{minipage}[t]{0.14\columnwidth}\centering\strut
0.6457\strut
\end{minipage}\tabularnewline
\begin{minipage}[t]{0.16\columnwidth}\centering\strut
200\strut
\end{minipage} & \begin{minipage}[t]{0.21\columnwidth}\centering\strut
0.0020\strut
\end{minipage} & \begin{minipage}[t]{0.20\columnwidth}\centering\strut
0.0010\strut
\end{minipage} & \begin{minipage}[t]{0.15\columnwidth}\centering\strut
0.8060\strut
\end{minipage} & \begin{minipage}[t]{0.14\columnwidth}\centering\strut
0.7218\strut
\end{minipage}\tabularnewline
\begin{minipage}[t]{0.16\columnwidth}\centering\strut
300\strut
\end{minipage} & \begin{minipage}[t]{0.21\columnwidth}\centering\strut
0.0020\strut
\end{minipage} & \begin{minipage}[t]{0.20\columnwidth}\centering\strut
0.0010\strut
\end{minipage} & \begin{minipage}[t]{0.15\columnwidth}\centering\strut
0.8134\strut
\end{minipage} & \begin{minipage}[t]{0.14\columnwidth}\centering\strut
0.7761\strut
\end{minipage}\tabularnewline
\bottomrule
\end{longtable}

** Classifer 2 - SVM **

\begin{longtable}[]{@{}ccccc@{}}
\toprule
\begin{minipage}[b]{0.16\columnwidth}\centering\strut
Training Set Size\strut
\end{minipage} & \begin{minipage}[b]{0.21\columnwidth}\centering\strut
Training Time\strut
\end{minipage} & \begin{minipage}[b]{0.20\columnwidth}\centering\strut
Prediction Time (test)\strut
\end{minipage} & \begin{minipage}[b]{0.15\columnwidth}\centering\strut
F1 Score (train)\strut
\end{minipage} & \begin{minipage}[b]{0.14\columnwidth}\centering\strut
F1 Score (test)\strut
\end{minipage}\tabularnewline
\midrule
\endhead
\begin{minipage}[t]{0.16\columnwidth}\centering\strut
100\strut
\end{minipage} & \begin{minipage}[t]{0.21\columnwidth}\centering\strut
0.0020\strut
\end{minipage} & \begin{minipage}[t]{0.20\columnwidth}\centering\strut
0.0020\strut
\end{minipage} & \begin{minipage}[t]{0.15\columnwidth}\centering\strut
0.8354\strut
\end{minipage} & \begin{minipage}[t]{0.14\columnwidth}\centering\strut
0.8025\strut
\end{minipage}\tabularnewline
\begin{minipage}[t]{0.16\columnwidth}\centering\strut
200\strut
\end{minipage} & \begin{minipage}[t]{0.21\columnwidth}\centering\strut
0.0080\strut
\end{minipage} & \begin{minipage}[t]{0.20\columnwidth}\centering\strut
0.0040\strut
\end{minipage} & \begin{minipage}[t]{0.15\columnwidth}\centering\strut
0.8431\strut
\end{minipage} & \begin{minipage}[t]{0.14\columnwidth}\centering\strut
0.8105\strut
\end{minipage}\tabularnewline
\begin{minipage}[t]{0.16\columnwidth}\centering\strut
300\strut
\end{minipage} & \begin{minipage}[t]{0.21\columnwidth}\centering\strut
0.0160\strut
\end{minipage} & \begin{minipage}[t]{0.20\columnwidth}\centering\strut
0.0090\strut
\end{minipage} & \begin{minipage}[t]{0.15\columnwidth}\centering\strut
0.8664\strut
\end{minipage} & \begin{minipage}[t]{0.14\columnwidth}\centering\strut
0.8052\strut
\end{minipage}\tabularnewline
\bottomrule
\end{longtable}

** Classifer 3 - Random Forest **

\begin{longtable}[]{@{}ccccc@{}}
\toprule
\begin{minipage}[b]{0.16\columnwidth}\centering\strut
Training Set Size\strut
\end{minipage} & \begin{minipage}[b]{0.21\columnwidth}\centering\strut
Training Time\strut
\end{minipage} & \begin{minipage}[b]{0.20\columnwidth}\centering\strut
Prediction Time (test)\strut
\end{minipage} & \begin{minipage}[b]{0.15\columnwidth}\centering\strut
F1 Score (train)\strut
\end{minipage} & \begin{minipage}[b]{0.14\columnwidth}\centering\strut
F1 Score (test)\strut
\end{minipage}\tabularnewline
\midrule
\endhead
\begin{minipage}[t]{0.16\columnwidth}\centering\strut
100\strut
\end{minipage} & \begin{minipage}[t]{0.21\columnwidth}\centering\strut
0.0250\strut
\end{minipage} & \begin{minipage}[t]{0.20\columnwidth}\centering\strut
0.0020\strut
\end{minipage} & \begin{minipage}[t]{0.15\columnwidth}\centering\strut
0.9924\strut
\end{minipage} & \begin{minipage}[t]{0.14\columnwidth}\centering\strut
0.7368\strut
\end{minipage}\tabularnewline
\begin{minipage}[t]{0.16\columnwidth}\centering\strut
200\strut
\end{minipage} & \begin{minipage}[t]{0.21\columnwidth}\centering\strut
0.0270\strut
\end{minipage} & \begin{minipage}[t]{0.20\columnwidth}\centering\strut
0.0020\strut
\end{minipage} & \begin{minipage}[t]{0.15\columnwidth}\centering\strut
0.9885\strut
\end{minipage} & \begin{minipage}[t]{0.14\columnwidth}\centering\strut
0.6822\strut
\end{minipage}\tabularnewline
\begin{minipage}[t]{0.16\columnwidth}\centering\strut
300\strut
\end{minipage} & \begin{minipage}[t]{0.21\columnwidth}\centering\strut
0.0320\strut
\end{minipage} & \begin{minipage}[t]{0.20\columnwidth}\centering\strut
0.0020\strut
\end{minipage} & \begin{minipage}[t]{0.15\columnwidth}\centering\strut
0.9975\strut
\end{minipage} & \begin{minipage}[t]{0.14\columnwidth}\centering\strut
0.7132\strut
\end{minipage}\tabularnewline
\bottomrule
\end{longtable}

    \subsection{Choosing the Best Model}\label{choosing-the-best-model}

In this final section, you will choose from the three supervised
learning models the \emph{best} model to use on the student data. You
will then perform a grid search optimization for the model over the
entire training set (\texttt{X\_train} and \texttt{y\_train}) by tuning
at least one parameter to improve upon the untuned model's F1 score.

    \subsubsection{Question 3 - Choosing the Best
Model}\label{question-3---choosing-the-best-model}

\emph{Based on the experiments you performed earlier, in one to two
paragraphs, explain to the board of supervisors what single model you
chose as the best model. Which model is generally the most appropriate
based on the available data, limited resources, cost, and performance?}

    \textbf{Answer: } First, i'll check how the model is performing based on
different train/test ratios, let's see what this will get me

    \begin{Verbatim}[commandchars=\\\{\}]
{\color{incolor}In [{\color{incolor}131}]:} \PY{k+kn}{import} \PY{n+nn}{matplotlib}\PY{n+nn}{.}\PY{n+nn}{pyplot} \PY{k}{as} \PY{n+nn}{plt}
          
          \PY{n}{step\PYZus{}size} \PY{o}{=} \PY{l+m+mi}{10}
          
          \PY{k}{def} \PY{n+nf}{get\PYZus{}F1}\PY{p}{(}\PY{n}{clf}\PY{p}{,} \PY{n}{training\PYZus{}set\PYZus{}size}\PY{p}{)}\PY{p}{:}
                  \PY{n}{X\PYZus{}train\PYZus{}sample} \PY{o}{=} \PY{n}{X\PYZus{}train}\PY{p}{[}\PY{p}{:}\PY{n}{training\PYZus{}set\PYZus{}size}\PY{p}{]}
                  \PY{n}{y\PYZus{}train\PYZus{}sample} \PY{o}{=} \PY{n}{y\PYZus{}train}\PY{p}{[}\PY{p}{:}\PY{n}{training\PYZus{}set\PYZus{}size}\PY{p}{]}
                  \PY{n}{train\PYZus{}classifier}\PY{p}{(}\PY{n}{clf}\PY{p}{,} \PY{n}{X\PYZus{}train\PYZus{}sample}\PY{p}{,} \PY{n}{y\PYZus{}train\PYZus{}sample}\PY{p}{)}
                  \PY{k}{return} \PY{n}{predict\PYZus{}labels}\PY{p}{(}\PY{n}{clf}\PY{p}{,} \PY{n}{X\PYZus{}test}\PY{p}{,} \PY{n}{y\PYZus{}test}\PY{p}{)}
              
          \PY{n}{f1\PYZus{}list\PYZus{}all} \PY{o}{=} \PY{p}{[}\PY{p}{]}
          \PY{k}{for} \PY{n}{clf} \PY{o+ow}{in} \PY{p}{[}\PY{n}{clf\PYZus{}A}\PY{p}{,} \PY{n}{clf\PYZus{}B}\PY{p}{,} \PY{n}{clf\PYZus{}C}\PY{p}{]}\PY{p}{:}
              \PY{n}{f1\PYZus{}list} \PY{o}{=} \PY{p}{[}\PY{p}{]}
              \PY{k}{for} \PY{n}{training\PYZus{}set\PYZus{}size} \PY{o+ow}{in} \PY{n+nb}{range}\PY{p}{(}\PY{l+m+mi}{50}\PY{p}{,}\PY{l+m+mi}{301}\PY{p}{,}\PY{n}{step\PYZus{}size}\PY{p}{)}\PY{p}{:}
                  \PY{n}{f1\PYZus{}list}\PY{o}{.}\PY{n}{append}\PY{p}{(}\PY{n}{get\PYZus{}F1}\PY{p}{(}\PY{n}{clf}\PY{p}{,} \PY{n}{training\PYZus{}set\PYZus{}size}\PY{p}{)}\PY{p}{)}
              \PY{n}{f1\PYZus{}list\PYZus{}all}\PY{o}{.}\PY{n}{append}\PY{p}{(}\PY{n}{f1\PYZus{}list}\PY{p}{)}
              
          \PY{c+c1}{\PYZsh{} Plot all lines, i changed the step for 1, 5 and 10 just to see the differences \PYZsh{}}
          \PY{n}{plt}\PY{o}{.}\PY{n}{plot}\PY{p}{(}\PY{n+nb}{range}\PY{p}{(}\PY{l+m+mi}{50}\PY{p}{,}\PY{l+m+mi}{301}\PY{p}{,}\PY{n}{step\PYZus{}size}\PY{p}{)}\PY{p}{,} \PY{n}{f1\PYZus{}list\PYZus{}all}\PY{p}{[}\PY{l+m+mi}{0}\PY{p}{]}\PY{p}{,} \PY{n}{label}\PY{o}{=}\PY{l+s+s1}{\PYZsq{}}\PY{l+s+s1}{Naive Bayes}\PY{l+s+s1}{\PYZsq{}}\PY{p}{)}
          \PY{n}{plt}\PY{o}{.}\PY{n}{plot}\PY{p}{(}\PY{n+nb}{range}\PY{p}{(}\PY{l+m+mi}{50}\PY{p}{,}\PY{l+m+mi}{301}\PY{p}{,}\PY{n}{step\PYZus{}size}\PY{p}{)}\PY{p}{,} \PY{n}{f1\PYZus{}list\PYZus{}all}\PY{p}{[}\PY{l+m+mi}{1}\PY{p}{]}\PY{p}{,} \PY{n}{label}\PY{o}{=}\PY{l+s+s1}{\PYZsq{}}\PY{l+s+s1}{SVM}\PY{l+s+s1}{\PYZsq{}}\PY{p}{)}
          \PY{n}{plt}\PY{o}{.}\PY{n}{plot}\PY{p}{(}\PY{n+nb}{range}\PY{p}{(}\PY{l+m+mi}{50}\PY{p}{,}\PY{l+m+mi}{301}\PY{p}{,}\PY{n}{step\PYZus{}size}\PY{p}{)}\PY{p}{,} \PY{n}{f1\PYZus{}list\PYZus{}all}\PY{p}{[}\PY{l+m+mi}{2}\PY{p}{]}\PY{p}{,} \PY{n}{label}\PY{o}{=}\PY{l+s+s1}{\PYZsq{}}\PY{l+s+s1}{Random Forest}\PY{l+s+s1}{\PYZsq{}}\PY{p}{)}
          
          \PY{c+c1}{\PYZsh{} Config axis \PYZsh{}}
          \PY{n}{plt}\PY{o}{.}\PY{n}{xlabel}\PY{p}{(}\PY{l+s+s1}{\PYZsq{}}\PY{l+s+s1}{Training set size}\PY{l+s+s1}{\PYZsq{}}\PY{p}{)}
          \PY{n}{plt}\PY{o}{.}\PY{n}{ylabel}\PY{p}{(}\PY{l+s+s1}{\PYZsq{}}\PY{l+s+s1}{F1 score}\PY{l+s+s1}{\PYZsq{}}\PY{p}{)}
          \PY{n}{plt}\PY{o}{.}\PY{n}{legend}\PY{p}{(}\PY{n}{loc}\PY{o}{=}\PY{l+m+mi}{4}\PY{p}{)}  
          
          \PY{c+c1}{\PYZsh{} Print  Chart \PYZsh{}}
          \PY{n}{plt}\PY{o}{.}\PY{n}{show}\PY{p}{(}\PY{p}{)}
\end{Verbatim}


    \begin{Verbatim}[commandchars=\\\{\}]
Trained model in 0.0010 seconds
Made predictions in 0.0020 seconds.
Trained model in 0.0010 seconds
Made predictions in 0.0010 seconds.
Trained model in 0.0010 seconds
Made predictions in 0.0000 seconds.
Trained model in 0.0000 seconds
Made predictions in 0.0000 seconds.
Trained model in 0.0020 seconds
Made predictions in 0.0010 seconds.
Trained model in 0.0030 seconds
Made predictions in 0.0010 seconds.
Trained model in 0.0020 seconds
Made predictions in 0.0000 seconds.
Trained model in 0.0020 seconds
Made predictions in 0.0010 seconds.
Trained model in 0.0020 seconds
Made predictions in 0.0010 seconds.
Trained model in 0.0020 seconds
Made predictions in 0.0000 seconds.
Trained model in 0.0020 seconds
Made predictions in 0.0010 seconds.
Trained model in 0.0030 seconds
Made predictions in 0.0010 seconds.
Trained model in 0.0020 seconds
Made predictions in 0.0010 seconds.
Trained model in 0.0020 seconds
Made predictions in 0.0000 seconds.
Trained model in 0.0010 seconds
Made predictions in 0.0010 seconds.
Trained model in 0.0010 seconds
Made predictions in 0.0000 seconds.
Trained model in 0.0020 seconds
Made predictions in 0.0000 seconds.
Trained model in 0.0010 seconds
Made predictions in 0.0010 seconds.
Trained model in 0.0010 seconds
Made predictions in 0.0010 seconds.
Trained model in 0.0010 seconds
Made predictions in 0.0000 seconds.
Trained model in 0.0010 seconds
Made predictions in 0.0000 seconds.
Trained model in 0.0020 seconds
Made predictions in 0.0000 seconds.
Trained model in 0.0010 seconds
Made predictions in 0.0010 seconds.
Trained model in 0.0010 seconds
Made predictions in 0.0010 seconds.
Trained model in 0.0010 seconds
Made predictions in 0.0010 seconds.
Trained model in 0.0010 seconds
Made predictions in 0.0000 seconds.
Trained model in 0.0010 seconds
Made predictions in 0.0010 seconds.
Trained model in 0.0010 seconds
Made predictions in 0.0010 seconds.
Trained model in 0.0010 seconds
Made predictions in 0.0010 seconds.
Trained model in 0.0010 seconds
Made predictions in 0.0010 seconds.
Trained model in 0.0020 seconds
Made predictions in 0.0000 seconds.
Trained model in 0.0010 seconds
Made predictions in 0.0010 seconds.
Trained model in 0.0020 seconds
Made predictions in 0.0010 seconds.
Trained model in 0.0020 seconds
Made predictions in 0.0010 seconds.
Trained model in 0.0020 seconds
Made predictions in 0.0010 seconds.
Trained model in 0.0030 seconds
Made predictions in 0.0010 seconds.
Trained model in 0.0030 seconds
Made predictions in 0.0010 seconds.
Trained model in 0.0030 seconds
Made predictions in 0.0020 seconds.
Trained model in 0.0030 seconds
Made predictions in 0.0020 seconds.
Trained model in 0.0040 seconds
Made predictions in 0.0010 seconds.
Trained model in 0.0040 seconds
Made predictions in 0.0010 seconds.
Trained model in 0.0040 seconds
Made predictions in 0.0020 seconds.
Trained model in 0.0050 seconds
Made predictions in 0.0020 seconds.
Trained model in 0.0050 seconds
Made predictions in 0.0010 seconds.
Trained model in 0.0050 seconds
Made predictions in 0.0020 seconds.
Trained model in 0.0230 seconds
Made predictions in 0.0030 seconds.
Trained model in 0.0070 seconds
Made predictions in 0.0010 seconds.
Trained model in 0.0090 seconds
Made predictions in 0.0140 seconds.
Trained model in 0.0110 seconds
Made predictions in 0.0020 seconds.
Trained model in 0.0080 seconds
Made predictions in 0.0020 seconds.
Trained model in 0.0080 seconds
Made predictions in 0.0020 seconds.
Trained model in 0.0090 seconds
Made predictions in 0.0030 seconds.
Trained model in 0.0270 seconds
Made predictions in 0.0010 seconds.
Trained model in 0.0250 seconds
Made predictions in 0.0010 seconds.
Trained model in 0.0260 seconds
Made predictions in 0.0020 seconds.
Trained model in 0.0270 seconds
Made predictions in 0.0010 seconds.
Trained model in 0.0250 seconds
Made predictions in 0.0010 seconds.
Trained model in 0.0240 seconds
Made predictions in 0.0020 seconds.
Trained model in 0.0240 seconds
Made predictions in 0.0010 seconds.
Trained model in 0.0260 seconds
Made predictions in 0.0010 seconds.
Trained model in 0.0250 seconds
Made predictions in 0.0010 seconds.
Trained model in 0.0260 seconds
Made predictions in 0.0010 seconds.
Trained model in 0.0270 seconds
Made predictions in 0.0020 seconds.
Trained model in 0.0280 seconds
Made predictions in 0.0010 seconds.
Trained model in 0.0350 seconds
Made predictions in 0.0220 seconds.
Trained model in 0.0570 seconds
Made predictions in 0.0050 seconds.
Trained model in 0.0640 seconds
Made predictions in 0.0040 seconds.
Trained model in 0.0560 seconds
Made predictions in 0.0040 seconds.
Trained model in 0.0550 seconds
Made predictions in 0.0020 seconds.
Trained model in 0.0520 seconds
Made predictions in 0.0040 seconds.
Trained model in 0.0560 seconds
Made predictions in 0.0020 seconds.
Trained model in 0.0580 seconds
Made predictions in 0.0040 seconds.
Trained model in 0.0760 seconds
Made predictions in 0.0030 seconds.
Trained model in 0.0690 seconds
Made predictions in 0.0060 seconds.
Trained model in 0.0440 seconds
Made predictions in 0.0010 seconds.
Trained model in 0.0440 seconds
Made predictions in 0.0020 seconds.
Trained model in 0.1040 seconds
Made predictions in 0.0050 seconds.
Trained model in 0.1090 seconds
Made predictions in 0.0090 seconds.

    \end{Verbatim}

    \begin{Verbatim}[commandchars=\\\{\}]
{\color{incolor}In [{\color{incolor}132}]:} \PY{n+nb}{print} \PY{l+s+s1}{\PYZsq{}}\PY{l+s+s1}{F1 mean for:}\PY{l+s+s1}{\PYZsq{}}
          \PY{n+nb}{print} \PY{l+s+s1}{\PYZsq{}}\PY{l+s+s1}{Naive Bayes : }\PY{l+s+si}{\PYZob{}:.4f\PYZcb{}}\PY{l+s+s1}{\PYZsq{}}\PY{o}{.}\PY{n}{format}\PY{p}{(}\PY{n}{np}\PY{o}{.}\PY{n}{mean}\PY{p}{(}\PY{n}{f1\PYZus{}list\PYZus{}all}\PY{p}{[}\PY{l+m+mi}{0}\PY{p}{]}\PY{p}{)}\PY{p}{)}
          \PY{n+nb}{print} \PY{l+s+s1}{\PYZsq{}}\PY{l+s+s1}{SVM (SVC) model : }\PY{l+s+si}{\PYZob{}:.4f\PYZcb{}}\PY{l+s+s1}{\PYZsq{}}\PY{o}{.}\PY{n}{format}\PY{p}{(}\PY{n}{np}\PY{o}{.}\PY{n}{mean}\PY{p}{(}\PY{n}{f1\PYZus{}list\PYZus{}all}\PY{p}{[}\PY{l+m+mi}{1}\PY{p}{]}\PY{p}{)}\PY{p}{)}
          \PY{n+nb}{print} \PY{l+s+s1}{\PYZsq{}}\PY{l+s+s1}{Random Forests model : }\PY{l+s+si}{\PYZob{}:.4f\PYZcb{}}\PY{l+s+s1}{\PYZsq{}}\PY{o}{.}\PY{n}{format}\PY{p}{(}\PY{n}{np}\PY{o}{.}\PY{n}{mean}\PY{p}{(}\PY{n}{f1\PYZus{}list\PYZus{}all}\PY{p}{[}\PY{l+m+mi}{2}\PY{p}{]}\PY{p}{)}\PY{p}{)}
\end{Verbatim}


    \begin{Verbatim}[commandchars=\\\{\}]
F1 mean for:
Naive Bayes : 0.6926
SVM (SVC) model : 0.8026
Random Forests model : 0.7210

    \end{Verbatim}

    Those are the charts and mean values for each step that i found

\begin{longtable}[]{@{}cccc@{}}
\toprule
Step Size & Naive Bayes & SVM & Random Forest\tabularnewline
\midrule
\endhead
1 & 0.6946 & 0.8021 & 0.7282\tabularnewline
5 & 0.6952 & 0.8023 & 0.7238\tabularnewline
10 & 0.6926 & 0.8026 & 0.7210\tabularnewline
\bottomrule
\end{longtable}

Indeed this doesn't vary much considering several train sets (step 1)
and generalizing using bigger steps

    \subsubsection{Step 1}\label{step-1}

\subsubsection{Step 5}\label{step-5}

\subsubsection{Step 10}\label{step-10}

    Using these charts, we can tell that the SVM is the model that has less
noise compared to the others. That, combined with its biggest f1 mean
for all the tests that i did on the test set, makes this model the one i
would chose for this particular problem, even knowing that the Naive
Bayes takes less time to run, in this particular problem that we have
such a small data set.

    \subsubsection{Question 4 - Model in Layman's
Terms}\label{question-4---model-in-laymans-terms}

\emph{In one to two paragraphs, explain to the board of directors in
layman's terms how the final model chosen is supposed to work. Be sure
that you are describing the major qualities of the model, such as how
the model is trained and how the model makes a prediction. Avoid using
advanced mathematical or technical jargon, such as describing equations
or discussing the algorithm implementation.}

    \textbf{Answer: } The model that we selected for this problem uses the
advantage of the clear separation between the classifiers to take its
guesses. For example, imagine that we want to separate 100 red balls
from 100 blue balls. Appart from the color, these balls are located in 2
different rooms. In the first scenario, we can see both rooms, so we
analyze that 90\% of the red balls are in the left room and 10\% of the
red balls in right room. A moment later, the light of the rooms is
turned off, a person comes to you from the left room and another from
the right room. Each one of them gives you a ball, and you must choose
the ball color without turning the lights on.

Because of the previous knowledge you had (when the lights were on) your
best guess is that the ball from the left room is red and the other one
is blue. You guess that because you have mentally drawn a line between
the rooms, using that separation to minimize the errors. Now imagine
that apart from that information, you know that the red balls tend to be
heavier and the blue balls tend to be lighter. Once you get them in your
hands, you can combine the information given (where they came plus their
weight) to improve even more your guess.

This is basically what this model does, it trace lines between the
parameters to separate the classes, once you have more parameters, it
traces planes between them (called hyperplanes), a simple representation
of this model is the following:

When we start adding more dimensions, the visualization can become
pretty complex, so this model is usualy represented in 2d or 3d for
better understandment.

    \subsubsection{Implementation: Model
Tuning}\label{implementation-model-tuning}

Fine tune the chosen model. Use grid search (\texttt{GridSearchCV}) with
at least one important parameter tuned with at least 3 different values.
You will need to use the entire training set for this. In the code cell
below, you will need to implement the following: - Import
\href{http://scikit-learn.org/stable/modules/generated/sklearn.grid_search.GridSearchCV.html}{\texttt{sklearn.grid\_search.gridSearchCV}}
and
\href{http://scikit-learn.org/stable/modules/generated/sklearn.metrics.make_scorer.html}{\texttt{sklearn.metrics.make\_scorer}}.
- Create a dictionary of parameters you wish to tune for the chosen
model. - Example:
\texttt{parameters\ =\ \{\textquotesingle{}parameter\textquotesingle{}\ :\ {[}list\ of\ values{]}\}}.
- Initialize the classifier you've chosen and store it in \texttt{clf}.
- Create the F1 scoring function using \texttt{make\_scorer} and store
it in \texttt{f1\_scorer}. - Set the \texttt{pos\_label} parameter to
the correct value! - Perform grid search on the classifier \texttt{clf}
using \texttt{f1\_scorer} as the scoring method, and store it in
\texttt{grid\_obj}. - Fit the grid search object to the training data
(\texttt{X\_train}, \texttt{y\_train}), and store it in
\texttt{grid\_obj}.

    \begin{Verbatim}[commandchars=\\\{\}]
{\color{incolor}In [{\color{incolor}135}]:} \PY{c+c1}{\PYZsh{} TODO: Import \PYZsq{}GridSearchCV\PYZsq{} and \PYZsq{}make\PYZus{}scorer\PYZsq{}}
          \PY{k+kn}{from} \PY{n+nn}{sklearn}\PY{n+nn}{.}\PY{n+nn}{grid\PYZus{}search} \PY{k}{import} \PY{n}{GridSearchCV} 
          \PY{k+kn}{from} \PY{n+nn}{sklearn}\PY{n+nn}{.}\PY{n+nn}{metrics} \PY{k}{import} \PY{n}{make\PYZus{}scorer}
          
          \PY{c+c1}{\PYZsh{} TODO: Create the parameters list you wish to tune}
          \PY{n}{parameters} \PY{o}{=} \PY{p}{\PYZob{}}\PY{l+s+s1}{\PYZsq{}}\PY{l+s+s1}{kernel}\PY{l+s+s1}{\PYZsq{}}\PY{p}{:}\PY{p}{[}\PY{l+s+s1}{\PYZsq{}}\PY{l+s+s1}{linear}\PY{l+s+s1}{\PYZsq{}}\PY{p}{,} \PY{l+s+s1}{\PYZsq{}}\PY{l+s+s1}{rbf}\PY{l+s+s1}{\PYZsq{}}\PY{p}{,} \PY{l+s+s1}{\PYZsq{}}\PY{l+s+s1}{poly}\PY{l+s+s1}{\PYZsq{}}\PY{p}{,} \PY{l+s+s1}{\PYZsq{}}\PY{l+s+s1}{sigmoid}\PY{l+s+s1}{\PYZsq{}}\PY{p}{]}\PY{p}{,} \PY{l+s+s2}{\PYZdq{}}\PY{l+s+s2}{C}\PY{l+s+s2}{\PYZdq{}}\PY{p}{:}\PY{p}{[}\PY{n+nb}{pow}\PY{p}{(}\PY{l+m+mi}{2}\PY{p}{,} \PY{n}{c}\PY{p}{)} \PY{k}{for} \PY{n}{c} \PY{o+ow}{in} \PY{n+nb}{range}\PY{p}{(}\PY{o}{\PYZhy{}}\PY{l+m+mi}{3}\PY{p}{,} \PY{l+m+mi}{3}\PY{p}{)}\PY{p}{]}\PY{p}{,} 
                        \PY{l+s+s2}{\PYZdq{}}\PY{l+s+s2}{gamma}\PY{l+s+s2}{\PYZdq{}}\PY{p}{:}\PY{p}{[}\PY{n+nb}{pow}\PY{p}{(}\PY{l+m+mi}{2}\PY{p}{,} \PY{n}{g}\PY{p}{)} \PY{k}{for} \PY{n}{g} \PY{o+ow}{in} \PY{n+nb}{range}\PY{p}{(}\PY{o}{\PYZhy{}}\PY{l+m+mi}{3}\PY{p}{,} \PY{l+m+mi}{3}\PY{p}{)}\PY{p}{]}\PY{p}{,} \PY{l+s+s2}{\PYZdq{}}\PY{l+s+s2}{degree}\PY{l+s+s2}{\PYZdq{}}\PY{p}{:}\PY{p}{[}\PY{n}{d} \PY{k}{for} \PY{n}{d} \PY{o+ow}{in} \PY{n+nb}{range}\PY{p}{(}\PY{l+m+mi}{1}\PY{p}{,} \PY{l+m+mi}{10}\PY{p}{)}\PY{p}{]}\PY{p}{\PYZcb{}}
          
          \PY{c+c1}{\PYZsh{} TODO: Initialize the classifier}
          \PY{n}{clf} \PY{o}{=} \PY{n}{SVC}\PY{p}{(}\PY{n}{random\PYZus{}state}\PY{o}{=}\PY{l+m+mi}{42}\PY{p}{)}
          
          \PY{c+c1}{\PYZsh{} TODO: Make an f1 scoring function using \PYZsq{}make\PYZus{}scorer\PYZsq{} }
          \PY{n}{f1\PYZus{}scorer} \PY{o}{=} \PY{n}{make\PYZus{}scorer}\PY{p}{(}\PY{n}{f1\PYZus{}score}\PY{p}{,} \PY{n}{pos\PYZus{}label}\PY{o}{=}\PY{l+s+s1}{\PYZsq{}}\PY{l+s+s1}{yes}\PY{l+s+s1}{\PYZsq{}}\PY{p}{)}
          
          \PY{c+c1}{\PYZsh{} TODO: Perform grid search on the classifier using the f1\PYZus{}scorer as the scoring method}
          \PY{n}{grid\PYZus{}obj} \PY{o}{=} \PY{n}{GridSearchCV}\PY{p}{(}\PY{n}{clf}\PY{p}{,} \PY{n}{param\PYZus{}grid}\PY{o}{=}\PY{n}{parameters}\PY{p}{,} \PY{n}{scoring}\PY{o}{=}\PY{n}{f1\PYZus{}scorer}\PY{p}{)}
          
          \PY{c+c1}{\PYZsh{} TODO: Fit the grid search object to the training data and find the optimal parameters}
          \PY{n}{grid\PYZus{}obj} \PY{o}{=} \PY{n}{grid\PYZus{}obj}\PY{o}{.}\PY{n}{fit}\PY{p}{(}\PY{n}{X\PYZus{}train}\PY{p}{,} \PY{n}{y\PYZus{}train}\PY{p}{)}
          
          \PY{c+c1}{\PYZsh{} Get the estimator}
          \PY{n}{clf} \PY{o}{=} \PY{n}{grid\PYZus{}obj}\PY{o}{.}\PY{n}{best\PYZus{}estimator\PYZus{}}
          
          \PY{c+c1}{\PYZsh{} Report the final F1 score for training and testing after parameter tuning}
          \PY{n+nb}{print} \PY{l+s+s2}{\PYZdq{}}\PY{l+s+s2}{Tuned model has a training F1 score of }\PY{l+s+si}{\PYZob{}:.4f\PYZcb{}}\PY{l+s+s2}{.}\PY{l+s+s2}{\PYZdq{}}\PY{o}{.}\PY{n}{format}\PY{p}{(}\PY{n}{predict\PYZus{}labels}\PY{p}{(}\PY{n}{clf}\PY{p}{,} \PY{n}{X\PYZus{}train}\PY{p}{,} \PY{n}{y\PYZus{}train}\PY{p}{)}\PY{p}{)}
          \PY{n+nb}{print} \PY{l+s+s2}{\PYZdq{}}\PY{l+s+s2}{Tuned model has a testing F1 score of }\PY{l+s+si}{\PYZob{}:.4f\PYZcb{}}\PY{l+s+s2}{.}\PY{l+s+s2}{\PYZdq{}}\PY{o}{.}\PY{n}{format}\PY{p}{(}\PY{n}{predict\PYZus{}labels}\PY{p}{(}\PY{n}{clf}\PY{p}{,} \PY{n}{X\PYZus{}test}\PY{p}{,} \PY{n}{y\PYZus{}test}\PY{p}{)}\PY{p}{)}
\end{Verbatim}


    \begin{Verbatim}[commandchars=\\\{\}]
Made predictions in 0.0050 seconds.
Tuned model has a training F1 score of 0.8319.
Made predictions in 0.0020 seconds.
Tuned model has a testing F1 score of 0.7947.

    \end{Verbatim}

    \subsubsection{Question 5 - Final F1
Score}\label{question-5---final-f1-score}

\emph{What is the final model's F1 score for training and testing? How
does that score compare to the untuned model?}

    \textbf{Answer: } With this tuned model, we got the following: *
Training F1 Score 0.8319 (previous was 0.8431) * Testing F1 Score 0.7947
(previous was 0.8105)

I have explored several hyperparameters values and some combinations as
a strategy of exaustive grid search so i can be sure i will get the best
fit possible for this dataset. This result show us a better model
against the non tuned one. The gap between the testing data and training
data tends to be smaller in better models. That means we managed to
avoid some of the overfitting that we had in the non tuned model.

    \begin{quote}
\textbf{Note}: Once you have completed all of the code implementations
and successfully answered each question above, you may finalize your
work by exporting the iPython Notebook as an HTML document. You can do
this by using the menu above and navigating to\\
\textbf{File -\textgreater{} Download as -\textgreater{} HTML (.html)}.
Include the finished document along with this notebook as your
submission.
\end{quote}


    % Add a bibliography block to the postdoc
    
    
    
    \end{document}
